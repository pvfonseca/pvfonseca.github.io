%-------------------------------------------------------------------------------
%	SECTION TITLE
%-------------------------------------------------------------------------------
\cvsection{Working Papers}


%-------------------------------------------------------------------------------
%	CONTENT
%-------------------------------------------------------------------------------
\begin{itemize}
    \item \textbf{Optimal monetary and fiscal policy with limited asset markets participation and government debt}

    \small{Building on a standard New Keynesian model, the model economy is augmented to incorporate the government's budget constraint -- where public expenditures are financed by distortionary taxation and/or issuing of long-term debt -- and the existence of limited asset markets participation. Without the ability to commit to an optimal plan, discretionary policies in the presence of government debt yield a state-dependent inflationary bias problem and also create a debt stabilization bias. Moreover, the presence of limited asset markets participation deepens distortions in the economy. As a result, the share's size of liquidity constrained agents impacts the long-run equilibrium values of relevant macroeconomic variables. Furthermore, the optimal response to shocks can be radically different under distinct levels of government debt  and fraction of rule-of-thumb consumers. Finally, higher levels of public debt causes a redistribution effect leading to rises in steady state inequalities among agents.}

    \item{\textbf{Strategic fiscal and monetary interactions in the Brazilian economy}

\small{This paper identifies the leadership structure of the game played by monetary and fiscal authorities in the Brazilian economy after the implementation of inflation targeting regime in 1999. A stylized small-scale New Keynesian model augmented with fiscal policy is estimated using Bayesian methods. I assume that monetary and fiscal authorities can act strategically under discretion in a non-cooperative setup and compare three different forms of games: (i) simultaneous move; (ii) fiscal leadership; and (iii) monetary leadership. I find strong empirical support for the hypothesis that the Brazilian fiscal authority acts as a Stackelberg leader. The results obtained can shed some light on the improvement of policy design in the Brazilian economy. \\ \emph{Submitted to Revista Brasileira de Economia (Brazilian Review of Economics).}}}

\item{\textbf{Strategic interactions, inflation conservatism and the level of government debt: A nonlinear analysis}

\small{This paper addresses the state dependencies in strategic interactions between an inflation conservative central bank and a benevolent fiscal authority. Building on a standard New Keynesian model extended to include fiscal policy and nominal government debt, I consider the effects of independent, discretionary and possibly non-cooperative policymakers, rather than joint optimal policies. The main contribution of this work is solving for the discretionary equilibrium of the policy games using nonlinear global solution techniques. I found that under a fiscal leadership policy game, delegating monetary policy to an inflation conservative authority can function as a device for fiscal discipline and reduces both stabilization and level inflationary biases. The consequences in terms of welfare losses are mostly harmless in this case. Nonetheless, in comparison to the cooperative setup, a simultaneous move policy game not only increases the gap between actual inflation and its target rate but is also associated with higher welfare losses. These losses are an increasing function of the degree of monetary conservatism.}}
\end{itemize}